\documentclass[twoside, twocolumn, 10pt]{article}
\usepackage{amsthm,amssymb,amsmath,amsfonts}
\usepackage{color}
\definecolor{r}{rgb}{0.5, 0.0, 0.7}
\usepackage{graphicx}
\usepackage[utf8]{inputenc}
\usepackage[spanish]{babel}
\def\ma{\mathbb}
\setlength{\topmargin}{-3cm} \addtolength{\textheight}{5.5cm}
\addtolength{\hoffset}{-1cm} \addtolength{\textwidth}{3cm}
\title{Ecuaciones Diferenciales Ordinarias I}
\author{P\'avel Ernesto Oropeza Alfaro \\Formulario 1}
\begin{document}
\date{}
\maketitle
\thispagestyle{empty}
{\begin{small}
\begin{enumerate}
    \item \textbf{Definición} y clasificación de las ecuaciones diferenciales. 
\begin{itemize}
    \item Es una ecuación que contiene derivadas de una o más 
        variables respecto a una o más variables independientes.
        $F(x,y, y^{\prime},y^{\prime},y^{(n)})=0$
    \item Se clasifican por \textbf{tipo}, \textbf{orden} y \textbf{linealidad}
    \item En la clasificación por orden, el orden de la ODE es el orden de la
        mayor derivada en la ecuación.
    \item Las ODE de primer orden a menudo se escriben como 
        $M(x,y)dx + N(x,y) dy = 0$
    \item Es común escribir las ODE de primer y segundo orden en su \textbf{forma normal}:
        $\frac{dy}{dx}=f(x,y)$ y $\frac{d^{2}y}{dx^{2}}=f(x,y,y^{\prime})$
\end{itemize}

\item \textbf{Solución de una ODE}. Es una función $\phi$ definida en un intervalo $I$ que
    tiene al menos $n$ derivadas continuas en $I$, las cuales al sustituirse en
    una ODE reducen la ecuación a una identidad.

\item \textbf{Factor Integrante $\mu(x)=e^{\int p(x) dx}$.} Una forma alternativa de escribir una ODE lineal 
    $\frac{dy}{dx}=f(x,y)$ es la siguiente:
    $y^{\prime} + p(x)y=g(x)$

Por ejemplo, $\frac{dy}{dx}=\frac{3}{2}-\frac{1y}{2}$ se puede escribir como:
$y^{\prime} + \frac{1y}{2}=\frac{3}{2}$
Para resolver una ODE lineal de primer orden se puede usar: 
\begin{center}
$y(x) = \frac{1}{\mu(x)}[ \int \mu(x)  g(x) dx + C]$  
\end{center}
\item \textbf{Clasificación de una ODE lineal de orden 1.}
\begin{itemize}
    \item Variables Separables: $\frac{dy}{dx}=\frac{g(x)}{h(y)}$
    \item Lineales: $y^{\prime} + p(x)y=g(x)$
    \item Exactas: $M(x,y)dx + N(x,y) dy = 0$
    \item Métodos especiales:
    \item [a)] Sustitución / Bernoulli
    \item [b)] Factor integrante
\end{itemize} 

\item \textbf{Ecuaciones lineales de Segundo Orden. (A partir de esta sección se emplea $x$ y $t$ como variables independientes en forma indistinta)} 
    \begin{center}
    $\frac{d^{2}y}{dx^{2}}=f(x,y,\frac{dy}{dx})$\\ 
    \end{center}

    Si $f$ puede escribirse como:

    \begin{center}
    $f(x,y,\frac{dy}{dx})=g(x)-p(x)\frac{dy}{dx}-q(x)y$, 
    \end{center}
    se dice que es lineal, donde
    $g,p$ y $q$ son funciones continuas en el $I$ solución.
    Reescribiendo y considerando la ecuación como homogénea ($g(x)=0$): 
    \begin{center}
    $p(x)y^{\prime\prime} + q(x)y^{\prime} + r(x)y = 0$
    \end{center}
    Los coeficientes $p,q$ y $r$ pueden ser constantes, en cuyo caso la ode se escribe como:
    \begin{equation} \label{coef-ctes}
     ay^{\prime\prime} + by^{\prime} + cy =0
    \end{equation}
Es una ode homogénea con coeficientes constantes, la cual puede resolverse con:
\begin{equation}
 y(t)=y_{1}(t) + y_{2}(t) 
\end{equation}
\begin{equation}
 y(t) = c_{1} e^{r_{1}t} + c_{2}e^{r_{2}t}
\end{equation}

\begin{itemize}
    \item Principio de superposición:
        Si $y_{1}$ y $y_{2}$ son soluciones por separado de 
        \begin{equation}
        L[y]=y^{\prime\prime} + p(t)y^{\prime} + q(t)y=0
        \end{equation}
        entonces la combinación lineal $c_{1}y_{1}(t)+c_{2}y_{2}(t)$ es también
        solución para cualquier valor de $c_{1}$ y $c_{2}$. A esto también hay que agregar que:
        
        \textit{El múltiplo constante $y=c_{1}y_{1}(t)$ de una solución $y_{1}(t)$ de una ode lineal homogénea, es también una solución.}
    \item Conjunto fundamental, independencial lineal y wronskiano.
    \begin{equation}
          W(y_{1}(t_{0}),y_{2}(t_{0})) = \begin{vmatrix}
                                          y_{1} & y_{2}\\
                                          y_{1}^{\prime} & y_{2}^{\prime}
                           \end{vmatrix}                           
    \end{equation}
    El valor de $W(y_{1}(t_{0}),y_{2}(t_{0}))$ debe ser \textbf{diferente} de cero, en un valor $t_{0}$ para considerar $y_{1}$ y $y_{2}$ un conjunto fundamental y poder construir
    la soluciones. 
    \item \textbf{Raíces complejas conjugadas.}
    A partir del polinomio característico
    $ar^{2} + br + c = 0$ al obtener las soluciones $r_{1}$ y $r_{2}$ se puede dar el caso de que sean imaginarias
    $r_{1} = a + ib$ y  $r_{2} = a - ib$
    En ese caso se emplea la Fórmula de Euler para \textit{eliminar} la parte imaginaria
    \begin{align*}
    e^{i\theta} = cos(\theta) + isin(\theta)\\
    e^{-i\theta} = cos(\theta) - isin(\theta)\\
    e^{a + ib} = e^{a}e^{ib}=e^{a}(cos(b) + i sin(b))    
    \end{align*}
    \item Ejemplo de uso de la fórmula de Euler:
    \begin{align*}
     2^{1-i} = e^{ln(2^{1-i})}=e^{(1-i)ln(2)}\\
     e^{(ln2 -iln2)}\\
     e^{ln2}(cos(ln2) - isin(ln2))\\
     2(cos(ln2) -isin(ln2))
    \end{align*}

    Al resolver \ref{coef-ctes}, la solución general es:
    \begin{equation}
    y(t) = c_{1}e^{(a+ib)t} + c_{2}e^{(a-ib)t}
    \end{equation}
    o bien, empleando la fórmula de Euler:
    \begin{equation}
     y(t) = e^{at}(c_{1} cos(bt) + c_{2} sin(bt))
    \end{equation}
    \item Raíces repetidas y Reducción de orden.
    
    Solución General:
    \begin{equation}
    y = c_{1} e^{r_{1}t} + c_{2} t e^{r_{2}t}
    \end{equation}
    
    Reducción de Orden: Hallar un método para encontrar soluciones que formen un $CF$
    \begin{equation}
    a_{2}(t)
     
    \end{equation}


    
    \item Ecuación de Euler-Cauchy:
    \begin{equation}
    t^{2}\frac{d^{2}y}{dt^{2}}+\alpha t\frac{dy}{dx} + \beta y = 0 
    \end{equation}

%     \item Condiciones iniciales. 
%     \item Dos ODE lineales de segundo orden típicas:
%     \begin{center}
%         $y^{\prime\prime}+5y^{\prime}+6y=0$ 
% 
%         $y^{\prime\prime}+\omega^{2}y=0$
%     \end{center}
% 
% 
%     \item Reducción de orden.
%     \item Gráficas de soluciones.
% 


\end{itemize} 
%\item \textbf{PRIMER PARCIAL}. 
% \item \textbf{Ecuaciones lineales de Segundo Orden (no homogéneas)}. 
% \begin{itemize}
%     \item Método 1. Proponer una solución particular adecuada
% \end{itemize}
\end{enumerate}
%
%\bigskip
%
 \vspace{.1cm} 
% \indent \textbf{Fecha de entrega:} Viernes 20 de Marzo de 2015. ( se puede entregar en equipos entre [1, 5] personas)
\end{small}}
\end{document}